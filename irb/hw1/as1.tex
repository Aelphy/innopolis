\documentclass[12pt]{article}

\usepackage[utf8]{inputenc}
\usepackage[T2A]{fontenc}
\usepackage[english,russian]{babel}
\usepackage{amssymb}
\usepackage{graphicx}
\graphicspath{ {images/} }

\textwidth=431pt
\textheight=600pt
\hoffset=-30pt
\voffset=-30pt

\usepackage{graphicx}
\usepackage{amsmath}
\makeatletter
\renewcommand{\@oddhead}{%
\vbox{%
\hbox to \textwidth{\strut \textit{Introduction to robotics, Assignment 1, Usvyatsov Mikhail} \hfill }
%\hbox to\textwidth{Лист\hfill Страница~\arabic{page}~из 2}
\hrule
\vspace{12pt}
}}
\renewcommand{\@oddfoot}{}
\makeatother


\begin{document}

%\tableofcontents

%\newpage

\begin{center}
\textbf{Assignment 1;\\
due Tuesday October 28}
\end{center}

\textbf{Part 1}	

\bigskip
	
\textbf{Exercise 1}		
		
\textbf{Solution}

\medskip

\newcounter{bcounter}
\begin{list}{(\alph{bcounter})~}{\usecounter{bcounter}}
\item 
R cannot be a rotation matrix because det(R) $\neq$ 1  
\item
$det(R^{-1}) = \dfrac{1}{det(R)} = -1$
So R cannot be the inverse of a rotation matrix
\item
$RR(RR)^T = RR R^T R^T = R I R^T = RR^T = I$

$det(RR) = det(R) * det(R) = 1$

So, RR can be a rotation matrix
\item
$2R  2R^T = 4 RR^T = 4I$

So, 2R cannot be a rotation matrix
\end{list}

\textbf{Exercise 2}		
		
\textbf{Solution}

\medskip

\begin{list}{(\alph{bcounter})~}{\usecounter{bcounter}}
\item 
R is yet undefined. General rotation matrix R is $R_Y(\alpha)R_Z(\beta)R_Y(\gamma)$ 
\item
$R1 = R_Y(\alpha)R_Z(\beta)R_Y(\gamma) =$

$\left( \begin{array}{ccc}
cos \alpha  & 0 & sin \alpha \\
0 & 1  & 0 \\
- sin \alpha & 0 & cos \alpha \end{array} \right)
\left( \begin{array}{ccc}
cos \beta  &  sin \alpha & 0 \\
sin \beta  &  0 & 0\\
0& cos \alpha & 1 \end{array} \right)
\left( \begin{array}{ccc}
cos \gamma  & 0 & sin \gamma \\
0 & 1  & 0 \\
- sin \gamma & 0 & cos \gamma\end{array} \right) = $

$\left( \begin{array}{ccc}
cos \alpha cos \beta cos \gamma - sin \alpha sin \gamma & - cos \alpha sin \beta & cos \gamma sin \alpha + sin \gamma cos \alpha cos \beta  \\
cos \gamma sin \beta & cos \beta & sin \beta sin \gamma \\
- cos \beta cos \gamma sin \alpha - cos \alpha sin \gamma & sin \alpha sin \beta & cos \alpha cos \gamma - cos \beta sin \alpha sin \gamma \end{array} \right)$

$R2 = R_Y(\hat{\alpha})R_Z(\hat{\beta}) =$

$\left( \begin{array}{ccc}
cos \hat{\alpha}  & 0 & sin \hat{\alpha} \\
0 & 1  & 0 \\
- sin \hat{\alpha} & 0 & cos \hat{\alpha}\end{array} \right)
\left( \begin{array}{ccc}
cos \hat{\beta}  &  sin \hat{\alpha} & 0 \\
sin \hat{\beta}  &  0 & 0\\
0& cos \hat{\alpha} & 1 \end{array} \right) = \left( \begin{array}{ccc}
cos \hat{\alpha} cos \hat{\beta} & - cos \hat{\alpha} sin \hat{\beta}  & sin \hat{\alpha} \\
sin \hat{\beta} & cos \hat{\beta} & 0 \\
- sin \hat{\alpha} cos \hat{\beta} & sin \hat{\alpha} sin \hat{\beta} & cos \hat{\alpha}\end{array} \right)$

We can see that $R2_{23}$ is zero, but $R1_{23}$ depends on $\hat{\beta}$ and $\hat{\gamma}$. So we can choose $\gamma$ and $\beta$ such, that $R1_{23}$ wouldn't be zero. Thus we cannot represent any rotation matrix by only to rotations.
\item
We remember that $R = R_Y(\alpha)R_Z(\beta)R_Y(\gamma) =$

$\left( \begin{array}{ccc}
cos \alpha cos \beta cos \gamma - sin \alpha sin \gamma & - cos \alpha sin \beta & cos \gamma sin \alpha + sin \gamma cos \alpha cos \beta  \\
cos \gamma sin \beta & cos \beta & sin \beta sin \gamma \\
- cos \beta cos \gamma sin \alpha - cos \alpha sin \gamma & sin \alpha sin \beta & cos \alpha cos \gamma - cos \beta sin \alpha sin \gamma \end{array} \right)$

And $R_Z(\hat{\beta})$ = $\left( \begin{array}{ccc}
cos \hat{\beta} & - sin \hat{\beta} & 0 \\
sin \hat{\beta} & cos \hat{\beta} & 0 \\
0 & 0 & 1 \end{array} \right)$

We can see that $R_{23} = sin \beta sin \gamma$ it is not equal to 1 always.
So, if $\gamma = - \alpha$ it is still false that $R = R_Z(\hat{\beta})$
\item
$R_Y(\hat{\alpha})$ = $\left( \begin{array}{ccc}
cos \hat{\alpha} & 0 & sin \hat{\alpha}\\
0 & 1 & 0 \\
- sin \hat{\alpha}& 0 & cos \hat{\alpha} \end{array} \right)$

We can see that $R_{23}$ from the previous example is equal to $sin \beta sin \gamma$ that is not zero always. So it is the fact that when $\alpha = - \beta$ $R \neq R_Y(\hat{\alpha})$. So, the assumption is false.
\end{list}

\textbf{Exercise 3}		
		
\textbf{Solution}

\medskip

First of all lets try to find all the eigenvalues.

We have to remember that $R = $

$\left( \begin{array}{ccc}
a_{11} & a_{12} & a_{13}  \\
a_{21} & a_{22} & a_{23} \\
a_{31} & a_{32} & a_{33} \end{array} \right)$

On the other hand, each column of this matrix is a rotation vector around each of axes.

Thus $X$ = $\left( \begin{array}{c}
a_{11} \\
a_{21} \\
a_{31} \end{array} \right)
Y = \left( \begin{array}{c}
a_{12} \\
a_{22} \\
a_{32} \end{array} \right)
Z = \left( \begin{array}{c}
a_{13} \\
a_{23} \\
a_{33} \end{array} \right)$

Moreover we know that:
$X = Y * Z$ 
$Y = Z * X$ 
$Z = X * Y$

So we can find that:

$a_{11} = a_{22} a_{33} - a_{32}a_{23}$

$a_{22} = a_{11} a_{33} - a_{13}a_{31}$ 

$a_{33} = a_{11} a_{22} - a_{12}a_{21}$ 

That are exactly some minors of matrix R. 

\medskip

We know that to find eigenvalues we have to solve the equation:

$\vert R - \lambda E \vert = 0$

So, $\vert R - \lambda E \vert = - (\lambda -1)[\lambda^2 - \lambda(a_{11} + a_{22} + a_{33} - 1) + 1] = 0$

According to [1], we have to let $cos \phi = \dfrac{a_{11}+ a_{22} + a_{33} - 1}{2}$. That is the only thing that is not clear from [1]. I still cannot show why it is so. However when we let that we can easily show that $\lambda_1 = 1$, $\lambda_{2,3}= cos \phi \pm i sin \phi$

Since we made previous conversions it is very easy to answer the questions.

\begin{list}{(\alph{bcounter})~}{\usecounter{bcounter}}
\item 
Wrong, because there are complex values in an answer
\item
True
\item
False because with some values of $\theta$ we can reach more 1.
\item
Due to the fact that we have complex values in an answer it is very easy to show that squared complex value is not always equals to 1.
\end{list}

\textbf{Exercise 4}		
		
\textbf{Solution}

\medskip

\begin{list}{(\alph{bcounter})~}{\usecounter{bcounter}}
\item 
No. X could be a vector, laying on the plane of rotation. And the rotation is 360 degree by orthogonal to the plane. 
\item
No. It could be a rotation axis.
\item
No. As mentioned in [a], it could lay on the plane of rotation if rotation is by 360 degrees of the orthogonal to the plane.  
\item
$(Rx)^T=x^T$

$x^T R^T = x^T$

$x^T R^T R = x^T R$

$x^T I = x^T R$

$x^T = x^T R$ QED
\end{list}

\textbf{Part 2}	

\bigskip

\textbf{Exercise 1}		
		
\textbf{Solution}

\medskip

$\left(\begin{array}{c}
^AP \\
1\end{array}\right) =  \left(\begin{array}{cccc}
\ & _{B}^AR & \ &^AP_{BORG} \\
0 & 0 & 0 & 1  \end{array}\right)\left(\begin{array}{c}
^BP \\
1\end{array}\right)$

$\left(\begin{array}{c}
^AP \\
1\end{array}\right) =  \left(\begin{array}{cccc}
1 & 0 & 0 & 4 \\
0 & 1 & 0 & -3 \\
0 & 0 & 1 & 0\\
0 & 0 & 0 & 1  \end{array}\right)\left(\begin{array}{c}
2 \\
6\\
0\\
1\end{array}\right)=\left(\begin{array}{c}
6 \\
3\\
0\\
1\end{array}\right)$

$^AP=\left(\begin{array}{c}
6 \\
3\\
0\end{array}\right)$

\medskip

\textbf{Exercise 2}		
		
\textbf{Solution}

\medskip

$\left(\begin{array}{c}
^AP \\
1\end{array}\right) =  \left(\begin{array}{cccc}
\ & _{B}^AR & \ &^AP_{BORG} \\
0 & 0 & 0 & 1  \end{array}\right)\left(\begin{array}{c}
^BP \\
1\end{array}\right)$

$\left(\begin{array}{c}
^AP \\
1\end{array}\right) =  \left(\begin{array}{cccc}
0 & -1 & 0 & 0 \\
1 & 0 & 0 & 0 \\
0 & 0 & 1 & 0 \\
0 & 0 & 0 & 1  \end{array}\right)\left(\begin{array}{c}
2 \\
6\\
0\\
1\end{array}\right)=\left(\begin{array}{c}
-6 \\
2 \\
0 \\
1\end{array}\right)$	

$^AP=\left(\begin{array}{c}
-6 \\
2\\
0\end{array}\right)$

\medskip

\textbf{Exercise 3}		
		
\textbf{Solution}

\medskip

\begin{list}{\alph{bcounter})~}{\usecounter{bcounter}}
\item
$_{1}^{0}T$

$^0 P_{BORG} = \left(4, 2, 0\right)$.

Rotations:

\newcounter{ccounter}
\begin{list}{\arabic{ccounter})~}{\usecounter{ccounter}}
\item Rotation around Y axis by $\Pi$ anti-clockwise.
\item Rotation around Z axis by $\dfrac{\Pi}{2}$ clockwise.
\end{list}

The rotation matrix is:
$R=\left(\begin{array}{ccc}
cos \Pi & 0 & sin \Pi \\
0 & 1 & 0 \\
-sin \Pi & 0 & cos \Pi \end{array}\right) \cdot 
\left(\begin{array}{ccc}
cos - \dfrac{\Pi}{2} & -sin - \dfrac{\Pi}{2} & 0 \\
sin - \dfrac{\Pi}{2} & cos - \dfrac{\Pi}{2} & 0 \\
0 & 0 & 1\end{array}\right) = 
\left(\begin{array}{ccc}
0 & -1 & 0 \\
-1 & 0 & 0 \\
0 & 0 & -1\end{array}\right)$

Hence, transformation matrix is:
$_{1}^{0}T = \left(\begin{array}{cccc}
0 & -1 & 0 & 4 \\
-1 & 0 & 0 & 2\\
0 & 0 & -1 & 0\\
0 & 0 & 0 & 1\end{array}\right)$

\item
$_{2}^1T$

$^2 P_{BORG} = \left(-4, 4, 0\right)$

Rotations:
\begin{list}{\arabic{ccounter})~}{\usecounter{ccounter}}
\item Rotation around X by $\dfrac{\pi}{4}$ clockwise.
\item Rotation around Z by $\pi$ anti-clockwise.
\item Rotation around Y by $\dfrac{\pi}{2}$ clockwise.
\end{list}

The rotation matrix is:

$R=\left(\begin{array}{ccc}
1 & 0 & 0 \\
0 & cos -\dfrac{\pi}{4} & -sin -\dfrac{\pi}{4}  \\
0 & sin -\dfrac{\pi}{4}  & cos -\dfrac{\pi}{4}\end{array}\right)\cdot
\left(\begin{array}{ccc}
cos \pi & -sin \pi & 0 \\
sin \pi & cos \pi & 0\\
0 & 0 & 1\end{array}\right)\cdot
\left(\begin{array}{ccc}
cos -\dfrac{\pi}{2} & 0 & sin -\dfrac{\pi}{2} \\
0 & 1 & 0\\
-sin -\dfrac{\pi}{2} & 0 & cos -\dfrac{\pi}{2} 
\end{array}\right)=
\left(\begin{array}{ccc}
0 & 0 & 1 \\
\dfrac{\sqrt{2}}{2} & -\dfrac{\sqrt{2}}{2} & 0 \\
\dfrac{\sqrt{2}}{2} & \dfrac{\sqrt{2}}{2} & 0\end{array}\right)$
 
Hence, transformation matrix is:
 
$_{2}^1T = \left(\begin{array}{cccc}
0 & 0 & 1 & -4 \\
\dfrac{\sqrt{2}}{2} & -\dfrac{\sqrt{2}}{2} & 0 & 4 \\
\dfrac{\sqrt{2}}{2} & \dfrac{\sqrt{2}}{2} & 0 & 0\\
0 & 0 & 0 & 1\end{array}\right)$

\item
$_{3}^2T$

$^2 P_{BORG} = \left(4, 5\sqrt{2}, \sqrt{2}\right)$

Rotations:
\begin{list}{\arabic{ccounter})~}{\usecounter{ccounter}}
\item Rotation around Z by $\dfrac{\Pi}{2}$ clockwise.
\item Rotation around X by $\dfrac{3 \Pi}{2}$ clockwise.
\end{list}
 
The rotation matrix is:

$R=\left(\begin{array}{ccc}
cos -\dfrac{\Pi}{2} & -sin -\dfrac{\Pi}{2} & 0 \\
sin -\dfrac{\Pi}{2} & cos -\dfrac{\Pi}{2} & 0\\
0 & 0 & 1\end{array}\right)\cdot 
\left(\begin{array}{ccc}
1 & 0 & 0 \\
0 & cos -\dfrac{3 \Pi}{2} & -sin -\dfrac{3 \Pi}{2}  \\
0 & sin -\dfrac{3 \Pi}{2} & cos -\dfrac{3 \Pi}{2} \end{array}\right)=
\left(\begin{array}{ccc}
0 & - \dfrac{\sqrt{2}}{2} & - \dfrac{\sqrt{2}}{2} \\
-1 & 0 & 0 \\
0 & \dfrac{\sqrt{2}}{2} & - \dfrac{\sqrt{2}}{2}\end{array}\right)$ 
\end{list}

Hence, transformation matrix is:

$_{2}^{3}T = \left(\begin{array}{cccc}
0 & - \dfrac{\sqrt{2}}{2} & -\dfrac{\sqrt{2}}{2} & 4 \\
-1 & 0 & 0 & 5\sqrt{2}\\
0 & \dfrac{\sqrt{2}}{2} & - \dfrac{\sqrt{2}}{2} & \sqrt{2}\\
0 & 0 & 0 & 1\end{array}\right)$

\newpage

\textbf{Exercise 4}		
		
\textbf{Solution}

\begin{list}{\alph{bcounter})~}{\usecounter{bcounter}}
\item
$_{1}^{0}T$

$^0 P_{BORG} = \left(0, 0, -9\right)$.

Rotations:

\begin{list}{\arabic{ccounter})~}{\usecounter{ccounter}}
\item
Rotation around Z axis by $\alpha^\circ$ clockwise.
\end{list}

The rotation matrix is:
$R=\left(\begin{array}{ccc}
cos\left(-\alpha\right) & -sin\left(-\alpha\right) & 0 \\
sin\left(-\alpha\right) & cos\left(-\alpha\right) & 0\\
0 & 0 & 1\end{array}\right)$

Hence, transformation matrix is:

$_{1}^{0}T = \left(\begin{array}{cccc}
cos\left(\alpha\right) & sin\left(\alpha\right) & 0 & 0 \\
-sin\left(\alpha\right) & cos\left(\alpha\right) & 0 & 0\\
0 & 0 & 1 & -9\\
0 & 0 & 0 & 1\end{array}\right)$

\item
$_{2}^{1}T$

$^1 P_{BORG} = \left(0, 0, -3\right)$.

Rotations:

\begin{list}{\arabic{ccounter})~}{\usecounter{ccounter}}
\item Rotation around X axis by $\alpha^\circ$ clockwise.
\item Rotation around Y axis by $\dfrac{\Pi}{2}$ clockwise.
\end{list}

The rotation matrix is:
$R=\left(\begin{array}{ccc}
cos\left(-\alpha\right) & -sin\left(-\alpha\right) & 0 \\
sin\left(-\alpha\right) & cos\left(-\alpha\right) & 0\\
0 & 0 & 1\end{array}\right)\cdot
\left(\begin{array}{ccc}
cos -\dfrac{\Pi}{2} & 0 & sin -\dfrac{\Pi}{2} \\
0 & 1 & 0\\
-sin -\dfrac{\Pi}{2} & 0 & cos -\dfrac{\Pi}{2} \end{array}\right) =\left(\begin{array}{ccc}
0 & sin\left(\alpha\right) & -cos\left(\alpha\right) \\
0 & cos\left(\alpha\right) & sin\left(\alpha\right)\\
1 & 0 & 0\end{array}\right)$

Hence, transformation matrix is:

$_{1}^{2}T = \left(\begin{array}{cccc}
0 & sin\left(\alpha\right)& -cos\left(\alpha\right) & 0\\
0 & cos\left(\alpha\right) & sin\left(\alpha\right) & 0\\
1 & 0 & 0 & -3 \\
0 & 0 & 0 & 1\end{array}\right)$

\end{list}

\newpage
\textbf{Exercise 5}		
		
\textbf{Solution}

\begin{list}{\alph{bcounter})~}{\usecounter{bcounter}}
\item
$_{1}^{0}T$

$^0 P_{BORG} = \left(3-3cos\left(\alpha\right), 2, -3sin\left(\alpha\right)\right)$.

Rotations:

\begin{list}{\arabic{ccounter})~}{\usecounter{ccounter}}
\item Rotation around Z axis by $\alpha^\circ + \dfrac{\Pi}{2}$ clockwise.
\item Rotation around X axis by $\dfrac{\Pi}{2}$ clockwise.
\end{list}

The rotation matrix is:

$R=\left(\begin{array}{ccc}
cos\left(-\left(\alpha+\dfrac{\Pi}{2}\right)\right) & -sin\left(-\left(\alpha+\dfrac{\Pi}{2}\right)\right) & 0 \\
sin\left(-\left(\alpha+\dfrac{\Pi}{2}\right)\right) & cos\left(-\left(\alpha+\dfrac{\Pi}{2}\right)\right) & 0\\
0 & 0 & 1\end{array}\right)\cdot 
\left(\begin{array}{ccc}
1 & 0 & 0\\
0 & cos -\dfrac{\Pi}{2} & -sin -\dfrac{\Pi}{2} \\
0 & sin -\dfrac{\Pi}{2} & cos -\dfrac{\Pi}{2} \end{array}\right] = 
\left(\begin{array}{ccc}
-sin\left(\alpha\right) & 0 & cos\left(\alpha\right)\\
- cos\left(\alpha\right) & 0 & -sin\left(\alpha\right)\\
0 & -1 & 0\end{array}\right)$

Hence, transformation matrix is:

$_{1}^{0}T = \left(\begin{array}{cccc}
-sin\left(\alpha\right) & 0 & cos\left(\alpha\right) & 3-3cos\left(\alpha\right)\\
- cos\left(\alpha\right) & 0 & - sin\left(\alpha\right) & 2\\
0 & -1 & 0 & -3sin\left(\alpha\right)\\
0 & 0 & 0 & 1 \end{array}\right)$

\newpage

\item
$_{2}^{1}T$

$^1 P_{BORG} = \left(0.2, 0, 1\right)$.

Rotations:

\begin{list}{\arabic{ccounter})~}{\usecounter{ccounter}}
\item Rotation around X axis by $\dfrac{\Pi}{2}$ anti-clockwise.

\end{list}

The rotation matrix is:

$R=
\left(\begin{array}{ccc}
1 & 0 & 0\\
0 & cos \dfrac{\Pi}{2} & -sin \dfrac{\Pi}{2} \\
0 & sin \dfrac{\Pi}{2} & cos \dfrac{\Pi}{2} \end{array}\right)
=
\left(\begin{array}{ccc}
1 & 0 & 0\\
0 & 0 & -1\\
0 & 1 & 0\end{array}\right)$

Hence, transformation matrix is:

$_{2}^{1}T = \left(\begin{array}{cccc}
1 & 0 & 0 & 0.2\\
0 & 0 & -1 & 0\\
0 & 1 & 0 & 1\\
0 & 0 & 0 & 1\end{array}\right)$

\end{list}

\medskip

\textbf{List of references}

[1] http://robotics.caltech.edu/~jwb/courses/ME115/handouts/rotation.pdf

\end{document}