\documentclass[12pt]{article}

\usepackage[utf8]{inputenc}
\usepackage[T2A]{fontenc}
\usepackage[english,russian]{babel}
\usepackage{amssymb}
\usepackage{graphicx}
\graphicspath{ {images/} }

\textwidth=431pt
\textheight=600pt
\hoffset=-15pt
\voffset=-15pt

\usepackage{graphicx}
\usepackage{amsmath}
\makeatletter
\renewcommand{\@oddhead}{%
\vbox{%
\hbox to \textwidth{\strut \textit{Introduction to robotics, Assignment 4, Markeeva Larisa, Usvyatsov Mikhail} \hfill }
%\hbox to\textwidth{Лист\hfill Страница~\arabic{page}~из 2}
\hrule
\vspace{12pt}
}}
\renewcommand{\@oddfoot}{}
\makeatother

\begin{document}

%\tableofcontents

%\newpage

\begin{center}
\textbf{Assignment 4;\\
due Tuesday December 2}
\end{center}

\textbf{Part 1}	

\bigskip
	
\textbf{Exercise 1}		
		
\textbf{Solution}

\medskip

\newcounter{bcounter}

It is obvious that $|^A\Omega_{A, B}| = |^B\Omega_{B, A}| $.
Due to the fact that when we talk about angular velocity we are not interested in translation of Frame B wrt Frame A, we can say that the differences between $^A\Omega_{A, B}$ and $^B\Omega_{B, A}$ is only in direction. They are opposite.

\begin{list}{(\alph{bcounter})~}{\usecounter{bcounter}}
\item False.
\item False. 
\item True.
\item False.
\end{list}

\textbf{Exercise 2}		
		
\textbf{Solution}

\medskip

Angular velocity is a vector that represents the axes of frame rotation. The length of this vector is the measure of speed of this rotation. The measure unit of rotation speed is $\dfrac{Radian}{sec}$

\begin{list}{(\alph{bcounter})~}{\usecounter{bcounter}}
\item False.
\item False. 
\item False.
\item True.
\end{list}

\textbf{Exercise 3}		
		
\textbf{Solution}

\medskip

According to the Wikipedia, Via Point is a point through which the robot's tool should pass without stopping; via points are programmed in order to move beyond obstacles or to bring the arm into a lower inertia posture for part of the motion. 

\begin{list}{(\alph{bcounter})~}{\usecounter{bcounter}}
\item True. We can define via point in order to improve path. 
\item True. According to [2] via points can be used in trajectory generation.
\item True. According to [1] via points are very useful to fit constraints of environment.
\item False. Via points cannot protect from target missing due to errors because visiting this points could be done with errors.
\end{list}

\textbf{Exercise 4}		
		
\textbf{Solution}

\medskip

\begin{list}{(\alph{bcounter})~}{\usecounter{bcounter}}
\item True. According to [3] in joint space we can represent schemes in low level polynomials, but in Cartesian space the formulas are mush more difficult and includes trigonometric.
\item True. According to [3] it works for situations without obstacles.
\item False. The support of via points is very difficult to calculate in Joint space according to [3].
\item False. It doesn't matter in what space to calculate the motion - the result will be the same. However, according to [3] joint space scheme is less accurate in Cartesian space.
\end{list}

\textbf{Part 2}	

\bigskip
	
\textbf{Exercise 1}		
		
\textbf{Solution}

\medskip

\textbf{Exercise 2}		
		
\textbf{Solution}

\medskip

\textbf{List of references}

[1] Introduction to Robotics: Module Trajectory generation and robot programming FH Darmstadt

[2] Task Space velocity Blending for RealTime Trajectory Generation

[3] A Texbook of Industrial Robotics, p. 169

\end{document}