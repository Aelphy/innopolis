\documentclass[8pt]{article}

\usepackage[utf8]{inputenc}
\usepackage[T2A]{fontenc}
\usepackage[english,russian]{babel}
\usepackage{amssymb}
\usepackage{graphicx}
\graphicspath{ {images/} }

\textwidth=431pt
\textheight=600pt
\hoffset=-15pt
\voffset=-15pt

\usepackage{graphicx}
\usepackage{amsmath}
\makeatletter
\renewcommand{\@oddhead}{%
\vbox{%
\hbox to \textwidth{\strut \textit{Introduction to robotics, Assignment 2, Markeeva Larisa, Usvyatsov Mikhail} \hfill }
%\hbox to\textwidth{Лист\hfill Страница~\arabic{page}~из 2}
\hrule
\vspace{12pt}
}}
\renewcommand{\@oddfoot}{}
\makeatother

\begin{document}

%\tableofcontents

%\newpage

\begin{center}
\textbf{Assignment 1;\\
due Friday November 7}
\end{center}

\textbf{Part 1}	

\bigskip
	
\textbf{Exercise 1}		
		
\textbf{Solution}

\medskip

The system is the closed kinematic chain.

\newcounter{bcounter}
\begin{list}{(\alph{bcounter})~}{\usecounter{bcounter}}
\item False
\item False
\item False
\item True.
\end{list}

\textbf{Exercise 2}		
		
\textbf{Solution}

\medskip

\begin{list}{(\alph{bcounter})~}{\usecounter{bcounter}}
\item FALSE. SCARA robots can consist only of 4 rotational axes
\item TRUE. 3 rotational axes can give us 3 DoF. The forth one cannot add one more.
\item FALSE. The Chebyshev linkage has 4 rotational joints and only one DoF. 
\item FALSE. The explanation is like in previous example.
\end{list}

\textbf{Exercise 3}		
		
\textbf{Solution}

\medskip

\begin{list}{(\alph{bcounter})~}{\usecounter{bcounter}}
\item
TRUE. By definition.
\item
FALSE. By definition.
\item
FALSE. Because of b)
\item
FALSE. Because of a)
\end{list}

\textbf{Exercise 4}		
		
\textbf{Solution}

\medskip

\begin{list}{(\alph{bcounter})~}{\usecounter{bcounter}}
\item
FALSE. We can choose coordinates frames in the end effector.
\item
TRUE. We can choose coordinates frames in the end effector, so it could be many matrices. If there is only one base frame there is only one DH matrix.
\item
FALSE. Because we can change direction of X and Z axes. However, multiplication of all DH matrices will give us translation from base frame to the frame in the end effector and it will be unique.
\item
FALSE. The same explanation as in c)
\end{list}

\textbf{Part 2}	

\bigskip

\textbf{Exercise 1}		
		
\textbf{Solution}

\medskip

DH :
$ \begin{array}{ccccc}
&a_{i-1} & \alpha_{i-1} & d_i & \theta_i \\
1&0 & 0 & 0 & \theta_1 \\
2&0 & - \dfrac{\pi}{2} & 0 & \theta_2 \\
3&20 & 0 & 0 & 0
\end{array} $

$^0_1T = \left( \begin{array}{cccc}
cos(\theta_1) & -sin(\theta_1) & 0 & 0 \\
sin(\theta_1) & cos(\theta_1) & 0 & 0 \\
0 & 0 & 1 & 0 \\
0 & 0 & 0 & 1
\end{array} \right) $

$^1_2T = \left( \begin{array}{cccc}
cos(\theta_2) & -sin(\theta_2) & 0 & 0 \\
0 & 0 & 1 & 0 \\
-sin(\theta_2) & -cos(\theta_2) & 0 & 0 \\
0 & 0 & 0 & 1
\end{array} \right) $

$^2_3T = \left( \begin{array}{cccc}
1 & 0 & 0 & 20 \\
0 & 1 & 0 & 0 \\
0& 0 & 1 & 0 \\
0 & 0 & 0 & 1
\end{array} \right) $

$^0_3T = \ ^0_1T ^1_2T ^2_3T = \left( \begin{array}{cccc}
cos(\theta_1) cos(\theta_2) & -cos(\theta_1) sin(\theta_2) & -sin(\theta_1) & 20cos(\theta_1) cos(\theta_2) \\
sin(\theta_1) cos(\theta_2) & - sin(\theta_1) sin(\theta_2) & cos(\theta_1) & 20 sin(\theta_1) cos(\theta_2) \\
-sin(\theta_2) & -cos(\theta_2) & 0 & -20 sin(\theta_2) \\
0 & 0 & 0 & 1
\end{array} \right) $

\textbf{Exercise 2}		
		
\textbf{Solution}

\medskip

\begin{list}{(\alph{bcounter})~}{\usecounter{bcounter}}
\item
DH :
$ \begin{array}{ccccc}
& a_{i-1} & \alpha_{i-1} & d_i & \theta_i \\
1& 0 & 0 & 0 & \theta_1 \\
2& 0.3 & \dfrac{\pi}{2} & 0 & \theta_2 \\
3& 1 & 0 & 0 & \theta_3 \\
4& 0.2 & - \dfrac{\pi}{2} & 0 & \theta_4 \\
5& 1.5 & 0 & 0 & \theta_5 \\
6& 0 & \dfrac{\pi}{2} & 0 & \theta_6
\end{array} $

$^0_1T = \left( \begin{array}{cccc}
cos(\theta_1) & -sin(\theta_1) & 0 & 0 \\
sin(\theta_1) & cos(\theta_1) & 0 & 0 \\
0 & 0 & 1 & 0 \\
0 & 0 & 0 & 1
\end{array} \right) $

$^1_2T = \left( \begin{array}{cccc}
cos(\theta_2) & -sin(\theta_2) & 0 & 0.3 \\
0 & 0 & -1 & 0 \\
sin(\theta_2) & cos(\theta_2) & 0 & 0 \\
0 & 0 & 0 & 1
\end{array} \right) $

$^2_3T = \left( \begin{array}{cccc}
cos(\theta_3) & -sin(\theta_3) & 0 & 1 \\
sin(\theta_3) & cos(\theta_3) & 0 & 0 \\
0 & 0 & 1 & 0 \\
0 & 0 & 0 & 1
\end{array} \right) $

$^3_4T = \left( \begin{array}{cccc}
cos(\theta_4) & -sin(\theta_4) & 0 & 0.2 \\
0 & 0 & 1 & 0 \\
-sin(\theta_4) & -cos(\theta_4) & 0 & 0 \\
0 & 0 & 0 & 1
\end{array} \right) $

$^4_5T = \left( \begin{array}{cccc}
cos(\theta_5) & -sin(\theta_5) & 0 & 1.5 \\
sin(\theta_5) & cos(\theta_5) & 0 & 0 \\
0 & 0 & 1 & 0 \\
0 & 0 & 0 & 1
\end{array} \right) $

$^5_6T = \left( \begin{array}{cccc}
cos(\theta_6) & -sin(\theta_6) & 0 & 0 \\
0 & 0 & 1 & 0 \\
sin(\theta_6) & cos(\theta_6) & 0 & 0 \\
0 & 0 & 0 & 1
\end{array} \right) $

$^0_6T = \ ^0_1T ^1_2T ^2_3T ^3_4T ^4_5T ^5_6T =$

\newcommand\scalemath[2]{\scalebox{#1}{\mbox{\ensuremath{\displaystyle #2}}}}

$ \left( 
\scalemath{0.38}{
\begin{array}{cccc}
- C_6 S_1 S_{4,5} + C_1 (C_{2,3} C_{4,5} C_6 - S_{2,3} S_6) & S_1 S_{4,5} S_6 - 
 C_1 (C_6 S_{2,3} + C_{2,3} C_{4,5} S_6) & -C_4 (C_5 S_1 + C_1 C_{2,3} S_5) + 
 S_4 (-C_1 C_{2,3} C_5 + S_1 S_5) & \dfrac{1}{10} (C_1 (3 + 2 C_{2,3} + 5 C_2 (2 + 3 C_3 C_4) - 
      15 C_4 S_2 S_3) - 15 S_1 S_4) \\
C_6 (C_5 (C_2 C_3 C_4 S_1 - 
       C_4 S_1 S_2 S_3 + 
       C_1 S_4) + (C_1 C_4 - 
       C_{2,3} S_1 S_4) S_5) - 
 S_1 S_{2,3} S_6 & -C_6 S_1 S_{2,3} - (C_5 (C_2 C_3 C_4 S_1 - 
       C_4 S_1 S_2 S_3 + 
       C_1 S_4) + (C_1 C_4 - 
       C_{2,3} S_1 S_4) S_5) S_6 & C_1 C_{4,5} - C_{2,3} S_1 S_{4,5} & \dfrac{1}{10} (S_1 (3 + 2 C_{2,3} + 5 C_2 (2 + 3 C_3 C_4) - 
      15 C_4 S_2 S_3) + 15 C_1 S_4) \\
 C_{4,5} C_{6} S_{2,3} + C_{2,3} S_6 & C_{2,3} C_6 - C_{4,5} S_{2,3} S_6 & -S_{2,3} S_{4,5} & S_2 + \dfrac{1}{10} (2 + 15 C_4) S_{2,3} \\
 0 & 0 & 0 & 1
\end{array}
}
 \right) $

\item

$^0_6T_1 = \left( \begin{array}{cccc}
0.1908 & -0.0065 & 0.9816 & 1.586 \\
-0.9816  & 0 & 0.1908 & -1.4724 \\
0.0012  & 1 & 0.0064 & 0.6034 \\
0 & 0 & 0 & 1
\end{array} \right) $

$^0_6T_2 = \left( \begin{array}{cccc}
-0.0678 & 0.8783 &  -0.4733 & -0.2295 \\
 0.2145 & 0.4761 & 0.8528 & 0.7092 \\
-0.9744  & 0.0437 & 0.2206 & -1.5369 \\
0 & 0 & 0 & 1
\end{array} \right) $

$^0_6T_3 = \left( \begin{array}{cccc}
-0.7272 & -0.1891 &  -0.6599 & -0.3338 \\
-0.6850 & 0.1387 & 0.7152 & 0.8347 \\
0.0437 & -0.9721 & 0.2304 & 0.4429 \\
0 & 0 & 0 & 1
\end{array} \right) $

$^0_6T_4 = \left( \begin{array}{cccc}
0.7857  & -0.2540  & -0.5641 &  -1.6624 \\
0.6163  &  0.2430  &  0.7491 &  0.2349 \\
0.0532  &  0.9362  &  -0.3475&  0.2124 \\
0 & 0 & 0 & 1
\end{array} \right) $
    
$^0_6T_5 = \left( \begin{array}{cccc}
  0.4057  &  0.3911 &  -0.8261 &  -0.1587 \\
 -0.5013  & -0.6606 &  -0.5589 &  1.8721 \\
 0.7643   & -0.6408 &  0.0719  & 0.9183 \\
0 & 0 & 0 & 1
\end{array} \right) $

\newpage

\textbf{Exercise 3}		
		
\textbf{Solution}



\newpage

\textbf{Exercise 4}		
		
\textbf{Solution}

DH :
$ \begin{array}{ccccc}
& a_{i-1} & \alpha_{i-1} & d_i & 0\\
1 & 0 & 0 & d_1 & 0 \\
2 & 0 & -\dfrac{\pi}{2} & d_2 & 0 \\
3 & 0 & 0 & -d_2 & \theta_3
\end{array} $

\textbf{Exercise 5}		
		
\textbf{Solution}

DH :
$ \begin{array}{ccccc}
& a_{i-1} & \alpha_{i-1} & d_i & 0\\
1 & 0 & 0 & 0 & \theta_1 \\
2 & 0 & -\dfrac{\pi}{2} & L_1 & \theta_2 \\
3 & 0 & -\dfrac{\pi}{2} & d_1 & 0
\end{array} $

\medskip

\end{list}

\end{document}