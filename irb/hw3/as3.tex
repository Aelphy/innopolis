\documentclass[12pt]{article}

\usepackage[utf8]{inputenc}
\usepackage[T2A]{fontenc}
\usepackage[english,russian]{babel}
\usepackage{amssymb}
\usepackage{graphicx}
\graphicspath{ {images/} }

\textwidth=431pt
\textheight=600pt
\hoffset=-15pt
\voffset=-15pt

\usepackage{graphicx}
\usepackage{amsmath}
\makeatletter
\renewcommand{\@oddhead}{%
\vbox{%
\hbox to \textwidth{\strut \textit{Introduction to robotics, Assignment 3, Markeeva Larisa, Usvyatsov Mikhail} \hfill }
%\hbox to\textwidth{Лист\hfill Страница~\arabic{page}~из 2}
\hrule
\vspace{12pt}
}}
\renewcommand{\@oddfoot}{}
\makeatother

\begin{document}

%\tableofcontents

%\newpage

\begin{center}
\textbf{Assignment 3;\\
due Wednesday November 11}
\end{center}

\textbf{Part 1}	

\bigskip
	
\textbf{Exercise 1}		
		
\textbf{Solution}

\medskip

\newcounter{bcounter}

According to [1], Manocha and Zhu (1994) proposed a generalized closed form solution which can be derived for 6 DOF (or less) kinematic chain.

\begin{list}{(\alph{bcounter})~}{\usecounter{bcounter}}
\item False. It is possible that the target for EE is unreachable.
\item False. It is possible that the target for EE is unreachable.
\item False. It is possible that the target for EE is unreachable.
\item True.
\end{list}

\textbf{Exercise 2}		
		
\textbf{Solution}

\medskip

\begin{list}{(\alph{bcounter})~}{\usecounter{bcounter}}
\item False. 3 DoF manipulator with rotation joints can have only 2 dimension work-space
\item False. Dextrous can be empty.
\item True. E.g. 1 DoF manipulator with 2 dimension work-space and rotation joint with different length of links.
\item False.
\end{list}

\textbf{Exercise 3}		
		
\textbf{Solution}

\medskip

\begin{list}{(\alph{bcounter})~}{\usecounter{bcounter}}
\item True.
\item False. Usually we go from trigonometric to transcendental equations.
\item True. We user FK during solving IK.
\item True. IK problem needs a very fast computational engine in order to make solution in real-time.
\end{list}

\textbf{Exercise 4}		
		
\textbf{Solution}

\medskip

\begin{list}{(\alph{bcounter})~}{\usecounter{bcounter}}
\item False. We have considered the case of revolute joints.
\item False. Links are not important in IK it can only affect the work-space.
\item False. It is solution for manipulators with 6DOF’s when three consecutive axis intersect.
\item False.
\end{list}

\textbf{Part 2}	

\bigskip
	
\textbf{Exercise 1}		
\newpage
		
\textbf{Solution}

DH:
\\
\begin{tabular}{|c|c|c|c|c|}
\hline
 & $a_{i-1}$ & $\alpha_{i-1}$ & $d_i$ & $\theta_i$ \\
 \hline
 1 & 0 & 0 & 0 & $\phi$\\
 \hline
 2 & 0 & $-\dfrac{\pi}{2}$ & 0 & $\theta$\\
 \hline
 3 & 0 & $\dfrac{\pi}{2}$ & -$\left(L+d+R_2\right)$ & 0\\
 \hline
\end{tabular}


Transformation matrices

\[^0_{1}T=\left[
\begin{array}{cccc}
cos\phi & -sin\phi & 0 & 0 \\
sin\phi & cos\phi & 0 & 0 \\
0 & 0 & 1 & 0\\
0 & 0 & 0 & 1
\end{array} \right]\]
\ 
\[^1_{2}T=\left[
\begin{array}{cccc}
cos\theta & -sin\theta & 0 & 0 \\
0 & 0 & 1 & 0\\
-sin\phi & -cos\phi & 0 & 0 \\
0 & 0 & 0 & 1
\end{array} \right]\]
\ 
\[^2_{3}T=\left[
\begin{array}{cccc}
1 & 0 & 0 & 0 \\
0 & 0 & -1 & L+d+R_2\\
0 & 1 & 0 & 0 \\
0 & 0 & 0 & 1
\end{array} \right]\]
\ 
\[^0_{3}T=\left[
\begin{array}{cccc}
cos\phi\cdot sin\theta & -sin\phi & cos\phi\cdot sin\theta & -\left(d+l+R_2\right)cos\phi\cdot sin\theta \\
sin\phi\cdot cos\theta & cos\phi & sin\phi\cdot sin\theta & -\left(d+l+R_2\right)sin\phi\cdot sin\theta\\
-sin\theta & 0 & cos\theta & -\left(d+l+R_2\right)cos\theta \\
0 & 0 & 0 & 1
\end{array} \right]\]

\[\left\{
\begin{array}{l}
  x = -\left(d+l+R_2\right)cos\phi\cdot sin\theta\\
  y = -\left(d+l+R_2\right)sin\phi\cdot sin\theta\\
  z =-\left(d+l+R_2\right)cos\theta
\end{array} \right.\]
\ 
\[\left\{
\begin{array}{l}
  y^2=-\left(d+l+R_2\right)^2 sin^2 \phi\cdot sin^2 \theta\\
  x^2+y^2 =  \left(d+l+R_2\right)^2 cos^2 \phi sin^2\theta + \left(d+l+R_2\right)^2sin^2 \phi sin^2 \theta = \left(d+l+R_2\right)^2 sin^2 \theta \\
  z^2=\left(d+l+R_2\right)^2 cos^2\theta
\end{array} \right.\]
\ 
\[\left\{
\begin{array}{l}
  y^2=-\left(d+l+R_2\right)^2 sin^2 \phi\cdot sin^2 \theta\\
  \dfrac{z^2}{\left(d+l+R_2\right)^2} = cos^2\theta\\
  x^2+y^2 =  \left(d+l+R_2\right)^2 \left(1-\dfrac{z^2}{\left(d+l+R_2\right)^2} \right) 
\end{array} \right.\]
\ 
\[\left\{
\begin{array}{l}
 d_1 = \sqrt{x^2+y^2+z^2}-l-R_2\\
 d_2 = -\sqrt{x^2+y^2+z^2}-l-R_2\\
  \dfrac{z^2}{\left(d+l+R_2\right)^2} = cos^2\theta\\
    y^2=-\left(d+l+R_2\right)^2 sin^2 \phi\cdot sin^2 \theta
\end{array} \right.\]
\ 
\[\left\{
\begin{array}{l}
 d_1 = \sqrt{x^2+y^2+z^2}-l-R_2\\
 d_2 = -\sqrt{x^2+y^2+z^2}-l-R_2\\
  \theta = arccos\left(\pm\sqrt{\dfrac{z}{\left(d+l+R_2\right)}}\right)\\
   y^2=-\left(d+l+R_2\right)^2 sin^2 \phi\cdot sin^2 \theta\\
\end{array} \right.\]
\ 
\[\left\{
\begin{array}{l}
 d_1 = \sqrt{x^2+y^2+z^2}-\left(l+R_2\right)\\
 d_2 = -\sqrt{x^2+y^2+z^2}-\left(l+R_2\right)\\
  \theta = arccos\left(\pm\dfrac{z}{d+l+R_2}\right)\\
   \phi = arcsin\left(\pm \dfrac{y\cdot sin\theta}{d+l+R_2}\right)\\
\end{array} \right.\]

\newpage

\textbf{Exercise 2}	
\newpage	
		
\textbf{Solution}

DH:
\\
\begin{tabular}{|c|c|c|c|c|}
\hline
 & $a_{i-1}$ & $\alpha_{i-1}$ & $d_i$ & $\theta_i$ \\
 \hline
 1 & 0 & 0 & $-L_1$ & 0\\
 \hline
 2 & 0 & $-\dfrac{\pi}{2}$ & 0 & $\theta$\\
 \hline
 3 & $-L2$ & $\dfrac{\pi}{2}$ & $p$ & $\phi$\\
 \hline
 4 & $n$ & 0 & 0 & 0\\
 \hline
\end{tabular}


Transformation matrices

\[^0_{1}T=\left[
\begin{array}{cccc}
1 & 0 & 0 & 0 \\
0 & 1 & 0 & 0 \\
0 & 0 & 1 & -L_1\\
0 & 0 & 0 & 1
\end{array} \right]\]
\ 
\[^1_{2}T=\left[
\begin{array}{cccc}
cos\theta & -sin\theta & 0 & 0 \\
0 & 0 & 1 & 0\\
-sin\theta & -cos\theta & 0 & 0 \\
0 & 0 & 0 & 1
\end{array} \right]\]
\ 
\[^2_{3}T=\left[
\begin{array}{cccc}
cos\phi & -sin\phi & 0 & -L_2 \\
0 & 0 & -1 & p\\
-sin\phi & -cos\phi & 0 & 0 \\
0 & 0 & 0 & 1
\end{array} \right]\]
\ 
\[^3_{4}T=\left[
\begin{array}{cccc}
1 & 0 & 0 & n \\
0 & 1 & 0 & 0\\
0 & 0 & 1 & 0 \\
0 & 0 & 0 & 1
\end{array} \right]\]
\

\[^0_{4}T=\left[
\begin{array}{cccc}
WE & WE& WE & ncos\theta\cdot cos\phi-L_2 cos\theta-psin\theta \\
WE & WE & WE & nsin\phi\\
WE & WE& WE & -nsin\theta\cdot cos\phi+L_2 sin\theta -pcos\theta -L_1 \\
0 & 0 & 0 & 1
\end{array} \right]\]
\ 
\[\left\{
\begin{array}{l}
\phi=arcsin\dfrac{y}{n}\\
 x = nsin\theta \cdot cos\phi - L_2 cos\theta -p\cdot sin\theta\\
 z=-nsin\theta\cdot cos\phi - L_2sit\theta -p\cdot cos\theta - L_1
\end{array} \right.\]
\ 
\[\left\{
\begin{array}{l}
\phi=arcsin\dfrac{y}{n}\\
 x = nsin\theta \cdot cos\phi - L_2 cos\theta -p\cdot sin\theta\\
 z=-nsin\theta\cdot cos\phi - L_2sit\theta -p\cdot  cos\theta - L_1\\
 u=tg\dfrac{\theta}{2}
\end{array} \right.\]
\ 
\[\left\{
\begin{array}{l}
\phi=arcsin\dfrac{y}{n}\\
 x = \dfrac{(1-u^2)(n\cdot cos\phi-L_2)-p\cdot 2u}{1+u^2}\\
 z=-nsin\theta\cdot cos\phi - L_2sit\theta -p\cdot  cos\theta - L_1\\
 u=tg\dfrac{\theta}{2}
\end{array} \right.\]
\ 
\[\left\{
\begin{array}{l}
\phi=arcsin\dfrac{y}{n}\\
 u_{1,2}=p\pm \sqrt{p^2-x^2+(ncos\phi-L_2)^2}\\
 \theta=2arcth(p\pm \sqrt{p^2-x^2+(ncos\phi-L_2)^2})\\
 L_1=L_2\cdot sin\theta -pcos\theta +nsin\theta cos\phi -z
\end{array} \right.\]

\textbf{Exercise 3}	
\newpage	
		
\textbf{Solution}

DH:
\\
\begin{tabular}{| c|c|c|c|c|}
\hline
 & $a_{i-1}$ & $\alpha_{i-1}$ & $d_i$ & $\theta_i$ \\
 \hline
 1 & 0 & 0 & $-p$ & 0\\
 \hline
 2 & 0 & $-\dfrac{\pi}{2}$ & $-\omega$ & 0\\
 \hline
 3 & 0 & 0 & 0 & $\theta$\\
 \hline
 4 & $r$ & 0 & 0 & 0\\
 \hline
\end{tabular}


Transformation matrices

\[^0_{1}T=\left[
\begin{array}{cccc}
1 & 0 & 0 & 0 \\
0 & 1 & 0 & 0 \\
0 & 0 & 1 & -p\\
0 & 0 & 0 & 1
\end{array} \right]\]
\
\[^1_{2}T=\left[
\begin{array}{cccc}
1 & 0 & 0 & 0 \\
0 & 0 & 1 & -\omega \\
0 & -1 & 0 & 0\\
0 & 0 & 0 & 1
\end{array} \right]\]
\
\[^2_{3}T=\left[
\begin{array}{cccc}
cos\theta & -sin\theta & 0 & 0 \\
sin\theta & cos\theta & 0 & 0 \\
0 & 0 & 1 & 0\\
0 & 0 & 0 & 1
\end{array} \right]\] 
\
\[^3_{4}T=\left[
\begin{array}{cccc}
1 & 0 & 0 & r \\
0 & 1 & 0 & 0 \\
0 & 0 & 1 & 0\\
0 & 0 & 0 & 1
\end{array} \right]\] 
\
\[^0_{4}T=\left[
\begin{array}{cccc}
cos\theta & -sin\theta & 0 & r \left(cos\theta+1\right) \\
0 & 0 & 1 & -w \\
-sin\theta & -cos\theta & 0 & -p-r\cdot sin\theta\\
0 & 0 & 0 & 1
\end{array} \right]\] 
\ 
\[\left\{
\begin{array}{l}
x=r\left(cos\theta+1\right)\\
y=-\omega\\
z=-r\cdot sin\theta-p
\end{array} \right.\]
\ 
\[\left\{
\begin{array}{l}
\omega=-y\\
\theta=arccos\left(\dfrac{x}{r}-1\right)\\
p=-\left(z+r\sqrt{1-\left(\dfrac{x}{r}-1\right)^2}\right)
\end{array} \right.\]
\newpage

\textbf{Exercise 4}	
\newpage	
		
\textbf{Solution}

DH:
\\
\begin{tabular}{| c|c|c|c|c|}
\hline
 & $a_{i-1}$ & $\alpha_{i-1}$ & $d_i$ & $\theta_i$ \\
 \hline
 1 & 0 & 0 & 0 & $\alpha$ \\
 \hline
 2 & 0 & $\dfrac{\pi}{2}$ & a & $\dfrac{\pi}{2}$\\
 \hline
 3 & 0 & 0 & -b & $\beta$\\
 \hline
 4 & 0 & $-\dfrac{\pi}{2}$ & 0 & $\gamma$ \\
 \hline
 5 & 0 & 0 & -c & 0
\end{tabular}

\medskip

Transformation matrices

\[^0_{1}T=\left[
\begin{array}{cccc}
cos\alpha & -sin \alpha & 0 & 0 \\
sin \alpha & cos \alpha & 0 & 0 \\
0 & 0 & 1 & 0 \\
0 & 0 & 0 & 1
\end{array} \right]\]
\
\[^1_{2}T=\left[
\begin{array}{cccc}
0 & -1 & 0 & 0 \\
0 & 0 & -1 & -a \\
1 & 0 & 0 & 0\\
0 & 0 & 0 & 1
\end{array} \right]\]
\
\[^2_{3}T=\left[
\begin{array}{cccc}
cos\beta & -sin\beta & 0 & 0 \\
sin\beta & cos\beta & 0 & 0 \\
0 & 0 & 1 & -b\\
0 & 0 & 0 & 1
\end{array} \right]\] 
\
\[^3_{4}T=\left[
\begin{array}{cccc}
cos \gamma & -sin \gamma & 0 & 0 \\
0 & 0 & 1 & 0 \\
- sin \gamma & - cos \gamma & 0 & 0\\
0 & 0 & 0 & 1
\end{array} \right]\] 
\
\[^4_{5}T=\left[
\begin{array}{cccc}
1 & 0 & 0 & 1 \\
0 & 1 & 0 & 0 \\
0 & 0 & 1 & -c\\
0 & 0 & 0 & 1
\end{array} \right]\] 
\
\[^0_{5}T=\left[
\begin{array}{cccc}
-c_{\alpha}s_{\beta}c_{\gamma} - s_{\alpha}s_{\gamma} & -c_{\alpha}s_{\beta}c_{\gamma} - s_{\alpha}c_{\gamma} & -c_{\alpha}c_{\beta} & -c_{\alpha}s_{\beta}c_{\gamma} - s_{\alpha}s_{\gamma} + c \cdot c_{\alpha}c_{\beta} +(a - b) \cdot s_{\alpha} \\
-s_{\alpha}s_{\beta}c_{\gamma} - c_{\alpha}s_{\gamma} & s_{\alpha}s_{\beta}s_{\gamma} + c_{\alpha}c_{\gamma} & -s_{\alpha}c_{\beta} & -s_{\alpha}s_{\beta}s_{\gamma} - c_{\alpha}s_{gamma} - c \cdot s_{\alpha}c_{\beta} + (b - a) \cdot c_{\alpha} \\
c_{\beta}c_{\gamma} & -c_{\beta}s_{\gamma} & -s_{\beta} & c_{\beta}c_{\gamma} + c \cdot s_{\beta} \\
0 & 0 & 0 & 1
\end{array} \right]\] 

\newpage

\textbf{Exercise 5}	
\newpage	
		
\textbf{Solution}

Here we will try geometric solution instead of algebraic.

$x_1 = x_0 - L_3 sin \phi$

$y_1 = y_0 - L_3 cos \phi$

$x_1^2 + y_1^2 = L_1^2 + L_2^2 - 2 L_1 L_2 cos \theta_2$

$\theta_2 = arccos \left( \dfrac{x_1^2 + y_1^2 - L_1^2 - L_2^2}{2 L_1 L_2} \right)$

$\gamma = \dfrac{\pi}{2} - \theta_1 - arctg(\dfrac{y_1}{x_1})$

$L_2^2 = L_1^2 + x_1^2 + y_1^2 - 2L_1 \sqrt{x_1^2 + y_1^2} cos \gamma$

$\theta_1 = arcsin \left( \dfrac{L_1^2 - L_2^2 + x_1^2 + y_1^2}{2L_1 \sqrt{x_1^2 + y_1^2}} \right) - arctg \left( \dfrac{y_1}{x_1} \right)$

$\theta_3 = \phi - \theta_1 - \theta_2$

\bigskip

\textbf{List of references}

[1] CLOSED FORM AND GENERALIZED INVERSE KINEMATIC SOLUTIONS FOR ANIMATING THE HUMAN ARTICULATED STRUCTURE. Kwan W. CHIN

\end{document}