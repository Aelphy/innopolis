 
\documentclass[11pt,a4paper]{article}

\usepackage[hidelinks,colorlinks=false]{hyperref}
\usepackage[titletoc,title]{appendix}
\usepackage[nottoc]{tocbibind}
\usepackage{graphicx}
\usepackage{makeidx}
\usepackage{setspace}
\usepackage[margin=1.0in]{geometry}
\usepackage{authblk}
\usepackage{rotating}
 
\begin{document}
 
\title{Final Year Thesis Topic:\\Development and implementation dynamic balance algorithms for bipedal robot locomotion.}
\author{Prof. Eugeni Magid\\\vspace{-4mm}Innopolis University\\e.magid@innopolis.ru}
\date{}

\maketitle

\section{Context and Background}

Human anatomy allows people to work in different workspaces. With the help of two legs human can swim, walk, and even climb the mountains. Such abilities are the reason of the progress in science and technology. However, there is huge amount of  applications where the human have to sacrifice his life and limb. Such works as nuclear objects control, car tests, submarine services, car driving, e.t.c. can be done only with double legged robots. Robots now are very restricted. For each particular problem we have a particular robot. Vacuum-cleaner robot cannot delete dust from the cupboard. But double legged mechanisms will be able to work with devices designed for people with precision of robot. 

\section{Problem Statement}

This project is concerned with analyzing the structure of bipedal robot, computer model design, developing the dynamic balance algorithm based on ZMP approach or any other for this robot and implementing it on the real mechanism.  

\section{Objectives and Possible Approaches}

The goal of this exercise is to develop a software system that can output commands for robot's motors and it should make the robot to reach the target position in minimal time. Commands define the angles for motors that affect the robot. To compute these angles the surveillance application needs to be able to do the following.
\begin{enumerate}
\item Measure ZMP point with help of pressure sensors.
\item Solve inverse kinematics problem.
\item Solve the system of differential equations representing dynamics.
\end{enumerate}
Computer model can be done in Simulink environment. It will allow to test the algorithm and improve it iteratively. Then it is necessary to check the model on the real robot and again iteratively improve the model. Finally, robot should be able to follow the strait line on the floor. 

\section{Tools and Resources}

This project with use the matlab, simulink environment, YARP/ROS. The final thesis will be written using \LaTeX \cite{Lamport86}.
 
\section{Plan}
\begin{enumerate}
	\item
		Compare different approaches to bipedal locomotion.
		
		Bipedal locomotion consists of several phases. There are walking and staying parts. Each of this part requires different type of stability. If robot is statically stable, then it wouldn't require any energy while it stays. On the other hand, during walking bipedal robot should be stable, but previous characteristic doesn't provide this property. So we came up with the idea of dynamical stability, that will allow the robot to move. 
		There are several ways to achieve it. E.g. neural networks, ZMP, passive walking, capture point, to name a only few.
	\item
		Choose the most appropriate approach.
		
		The process of bipedal walking consists of continuous falling of the robot but it have to prevent it on time and change the phase. From this point of view it doesn't seems that neural networks are the good idea because we can describe the model precisely. More over bipedal robot is interesting because of its anthropomorphism because of it robot can be placed in the same environment as a human. This requirements make us think about the approach that is energy and computationally efficient and also provide a good precision of walking in several degrees of uncertainty.  
	\item
		Make the computer model of the robot.
		
		During the development we should compare several models ща the control, also mistakes are inevitable. Thus virtual simulation environment is necessary to prevent robot failure and reduce the time of problems identifying.
	\item
		Implement algorithm on this model.
		
		Algorithm should be implemented in simulator to prove that it allows control the entire model.  
	\item
		Iteratively improve the model and algorithm.
		
		It is inevitable that some approaches would be better than others, therefore  	
		it is very important to improve the model iteratively, and so to achieve better solution of the problem. 
	\item
		Implement the algorithm on practice.
		
		Theoretical solving of problem cannot guarantee that it really works. So real world test would be the best prove, the the work is done correctly.
	\item
		Adapt the model.
		
		After constructing the simple model, we want to make it more complicated to be able to work in different conditions and environments and solve real world problems. 
\end{enumerate}

\newpage

\section{Literature review}

Bipedal locomotion is a very complex task. It still doesn't have complete general solution however the research of this are has a long history. The development of the models starts from the inverted pendulum model of human walking and goes to the complex approach of actuated passive walking with ZMP control.\\
According to \cite{zmp_dynamic_walking} we can divide all the humanoid robots into two big groups: ZMP-controlled ones and passive - dynamic walkers.\\
Miomir Vukobratovic in \cite{zmp_vuko} defines ZMP as a point in which we can reduce all the forces with one single force. This ZMP point should be on the foot. The problem is that we cannot manipulate the foot directly. According to \cite{zmp_vuko} we can do it by ensuring the appropriate dynamics of the mechanism above the foot. If the resulting force in ZMP lies not in vertical direction than foot will slide. It means that dynamical stability was not achieved due to the fact, that there is a rotational moment that will affect the robot. On the other hand, if ZMP was achieved in the polygon of foot and moreover it coincides with the contact point, than robot is stable, due to the fact, that all the resulting forces lies in vertical direction. During the walk the position of ZMP should be computed simultaneously and the problem of control is to keep ZMP and contact point to be coincided.\\
In \cite{zmp_dynamic_walking} it was mentioned that ZMP approach give us the solution that is based on the principle of dynamical stability, however it is not energy efficient. It requires simultaneous control over all the joints of the robot. The method that was described in \cite{passive_walker} is called passive-walker dynamics and it uses gravity forces to reduce the amount of necessary energy to control the robot.\\
It was mentioned earlier that active control of the robot should be performed with applying dynamical stability principle, elsewhere the robot will loose the balance and fall. So, it makes sense to apply passive-walker dynamics with ZMP based control. According to \cite{zmp_vuko} ZMP method is the most well known and so it is necessary to start with it. Than iteratively increasing the complexity of the problem and applying new technics and constraints we will reach the optimal solution.\\
According to \cite{zmp_vuko} the most important task in the bipedal locomotion is to maintain dynamical stability. It can be accomplished if the foot have a full contact with the ground, it means, that the contact is not only in the edge or in the point. Moreover it shows that ZMP position depends on the robot dynamics: the resulting force in the contact polygon and total moment there. So, during the motion the position of ZMP changes and there are border situations when ZMP reaches the edge of support polygon. In these situations if additional moments appear, robot will rotate around foot edge and collapse. \cite{zmp_vuko} suggests the way to measure the load on the sole via force sensors on it. The algorithm of ZMP control is quite obvious. Compute wanted ZMP coordinates, measure the error and apply correcting signal. Very important notion is about Center of Pressure (CoP). According to \cite{zmp_vuko} the pressure between foot and the ground can be replaced with the force applied in CoP. With this we can define stability condition as ZMP and CoP coincident.

\newpage

\begin{thebibliography}{2}
	\bibitem{Lamport86}{
		Lesley Lamport. LATEX: A Document Preparation System.
		\emph{Addison-Wesley, Reading, Massachusetts, 1986} 
	}
	\bibitem{zmp_eind}{
		Dekker M.H.P. Zero-moment point method for stable biped walking. 
		\emph{Eindhoven University of Technology, 2009}
	}
	\bibitem{zmp_hanyang}{
		Jong H. Park., Yong K. Rhee. ZMP Trajectory Generation for Reduced Trunk Motions of Biped Robots. 
		\emph{Intelligent Robots and Systems, 1998}
	}
	\bibitem{zmp_dynamic_walking}{
		Manchester I. R. et al. Stable dynamic walking over uneven terrain. 
		\emph{The International Journal of Robotics Research, 2011}
	}
	\bibitem{zmp_vuko}{
		Vukobratovic M., Borovac B. Zero-moment point-thirty five years of its life.
		\emph{International Journal of Humanoid Robotics, 2004}
	}
	\bibitem{zmp_eind2}{
		van Zutven P. W. M. Modeling, identification and stability of humanoid robots.
		\emph{Eindhoven University of Technology, 2009}
	}
	\bibitem{passive_walker}{
		Collins S. H., Wisse M., Ruina A. A three-dimensional passive-dynamic walking robot with two legs and knees.
		\emph{The International Journal of Robotics Research, 2001}
	}
\end{thebibliography}

\bibliography{robotics}
  
\end{document}

