 
\documentclass[11pt,a4paper]{article}

\usepackage[hidelinks,colorlinks=false]{hyperref}
\usepackage[titletoc,title]{appendix}
\usepackage[nottoc]{tocbibind}
\usepackage{graphicx}
\usepackage{makeidx}
\usepackage{setspace}
\usepackage[margin=1.0in]{geometry}
\usepackage{authblk}
\usepackage{rotating}
 
\begin{document}
 
\title{Final Year Thesis Topic:\\Development and implementation dynamic balance algorithms for bipedal robot locomotion.}
\author{Prof. Eugeni Magid\\\vspace{-4mm}Innopolis University\\e.magid@innopolis.ru}
\date{}

\maketitle

\section{Context and Background}

Human anatomy allows people to work in different workspaces. With the help of two legs human can swim, walk, and even climb the mountains. Such abilities are the reason of the progress in science and technology. However, there is huge amount of  applications where the human have to sacrifice his life and limb. Such works as nuclear objects control, car tests, submarine services, car driving, e.t.c. can be done only with double legged robots. Robots now are very restricted. For each particular problem we have a particular robot. Vacuum-cleaner robot cannot delete dust from the cupboard. But double legged mechanisms will be able to work with devices designed for people with precision of robot. 

\section{Problem Statement}

This project is concerned with analyzing the structure of bipedal robot, computer model design, developing the dynamic balance algorithm based on ZMP approach or any other for this robot and implementing it on the real mechanism.  

\section{Objectives and Possible Approaches}

The goal of this exercise is to develop a software system that can output commands for robot's motors and it should make the robot to reach the target position in minimal time. Commands define the angles for motors that affect the robot. To compute these angles the surveillance application needs to be able to do the following.
\begin{enumerate}
\item Measure ZMP point with help of pressure sensors.
\item Solve inverse kinematics problem.
\item Solve the system of differential equations representing dynamics.
\end{enumerate}
Computer model can be done in Simulink environment. It will allow to test the algorithm and improve it iteratively. Then it is necessary to check the model on the real robot and again iteratively improve the model. Finally, robot should be able to follow the strait line on the floor. 

\section{Tools and Resources}

This project with use the matlab, simulink environment, YARP/ROS. The final thesis will be written using \LaTeX \cite{Lamport86}.
 
\section{Plan}
\begin{enumerate}
	\item
		Compare different approaches to bipedal locomotion.
		
		Bipedal locomotion consists of several phases. There are walking and staying parts. Each of this part requires different type of stability. If robot is statically stable, then it wouldn't require any energy while it stays. On the other hand, during walking bipedal robot should be stable, but previous characteristic doesn't provide this property. So we came up with the idea of dynamical stability, that will allow the robot to move. 
		There are several ways to achieve it. E.g. neural networks, ZMP, passive walking, capture point, to name a only few.
	\item
		Choose the most appropriate approach.
		
		The process of bipedal walking consists of continuous falling of the robot but it have to prevent it on time and change the phase. From this point of view it doesn't seems that neural networks are the good idea because we can describe the model precisely. More over bipedal robot is interesting because of its anthropomorphism because of it robot can be placed in the same environment as a human. This requirements make us think about the approach that is energy and computationally efficient and also provide a good precision of walking in several degrees of uncertainty.  
	\item
		Make the computer model of the robot.
		
		During the development we should compare several models ща the control, also mistakes are inevitable. Thus virtual simulation environment is necessary to prevent robot failure and reduce the time of problems identifying.
	\item
		Implement algorithm on this model.
		
		Algorithm should be implemented in simulator to prove that it allows control the entire model.  
	\item
		Iteratively improve the model and algorithm.
		
		It is inevitable that some approaches would be better than others, therefore  	
		it is very important to improve the model iteratively, and so to achieve better solution of the problem. 
	\item
		Implement the algorithm on practice.
		
		Theoretical solving of problem cannot guarantee that it really works. So real world test would be the best prove, the the work is done correctly.
	\item
		Adapt the model.
		
		After constructing the simple model, we want to make it more complicated to be able to work in different conditions and environments and solve real world problems. 
\end{enumerate}

\bibliographystyle{unsrt}
 
%\bibliography{robotics}
  
\end{document}

