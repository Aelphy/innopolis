 \documentclass[12pt]{article}
\usepackage[T2A]{fontenc}
\usepackage[utf8]{inputenc}        % Кодировка входного документа;
                                    % при необходимости, вместо cp1251
                                    % можно указать cp866 (Alt-кодировка
                                    % DOS) или koi8-r.

\usepackage[english,russian]{babel} % Включение русификации, русских и
                                    % английских стилей и переносов
%%\usepackage{a4}
%%\usepackage{moreverb}
\usepackage{amsmath,amsfonts,amsthm,amssymb,amsbsy,amstext,amscd,amsxtra,multicol}
\usepackage{verbatim}
\usepackage{tikz} %Рисование автоматов
\usetikzlibrary{automata,positioning, graphs}
\usepackage{multicol} %Несколько колонок
\usepackage{graphicx}
\usepackage{mathdots}
\usepackage[colorlinks,urlcolor=blue]{hyperref}
\usepackage[stable]{footmisc}

%% \voffset-5mm
%% \def\baselinestretch{1.44}
\renewcommand{\theequation}{\arabic{equation}}
\def\hm#1{#1\nobreak\discretionary{}{\hbox{$#1$}}{}}
\newtheorem{Lemma}{Лемма}
\theoremstyle{definiton}
\newtheorem{Remark}{Замечание}
%%\newtheorem{Def}{Определение}
\newtheorem{Claim}{Утверждение}
\newtheorem{Cor}{Следствие}
\newtheorem{Theorem}{Теорема}
\theoremstyle{definition}
\newtheorem{Example}{Пример}
\newtheorem*{known}{Теорема}
\def\proofname{Доказательство}
\theoremstyle{definition}
\newtheorem{Def}{Определение}



%\date{22 июня 2011 г.}
\let\leq\leqslant
\let\geq\geqslant
\def\MT{\mathrm{MT}}
%Обозначения ``ажуром''
\def\BB{\mathbb B}
\def\CC{\mathbb C}
\def\RR{\mathbb R}
\def\SS{\mathbb S}
\def\ZZ{\mathbb Z}
\def\NN{\mathbb N}
\def\FF{\mathbb F}
%греческие буквы
\let\epsilon\varepsilon
\let\es\emptyset
\let\eps\varepsilon
\let\al\alpha
\let\sg\sigma
\let\ga\gamma
\let\ph\varphi
\let\om\omega
\let\ld\lambda
\let\Ld\Lambda
\let\vk\varkappa
\let\Om\Omega
\def\abstractname{}

\def\R{{\cal R}}
\def\A{{\cal A}}
\def\B{{\cal B}}
\def\C{{\cal C}}
\def\D{{\cal D}}
\let\w\omega

%классы сложности
\def\REG{{\mathsf{REG}}}
\def\CFL{{\mathsf{CFL}}}
\newcounter{problem}
\newcounter{uproblem}
\newcounter{subproblem}
\def\pr{\medskip\noindent\stepcounter{problem}{\bf \theproblem .  }\setcounter{subproblem}{0}}
\def\prstar{\medskip\noindent\stepcounter{problem}{\bf $\theproblem^*$\negthickspace.  }\setcounter{subproblem}{0} }
\def\prpfrom[#1]{\medskip\noindent\stepcounter{problem}{\bf Задача \theproblem~(№#1 из задания).  }\setcounter{subproblem}{0} }
\def\prp{\medskip\noindent\stepcounter{problem}{\bf Задача \theproblem .  }\setcounter{subproblem}{0} }
\def\prpstar{\medskip\noindent\stepcounter{problem}{\bf Задача $\bf\theproblem^*$\negthickspace.  }\setcounter{subproblem}{0} }
\def\prdag{\medskip\noindent\stepcounter{problem}{\bf Задача $\theproblem^{^\dagger}$\negthickspace\,.  }\setcounter{subproblem}{0} }
\def\upr{\medskip\noindent\stepcounter{uproblem}{\bf Упражнение \theuproblem .  }\setcounter{subproblem}{0} }
%\def\prp{\vspace{5pt}\stepcounter{problem}{\bf Задача \theproblem .  } }
%\def\prs{\vspace{5pt}\stepcounter{problem}{\bf \theproblem .*   }
\def\prsub{\medskip\noindent\stepcounter{subproblem}{\rm \thesubproblem .  } }
\def\prsubstar{\medskip\noindent\stepcounter{subproblem}{\rm $\thesubproblem^*$\negthickspace.  } }
%прочее
\def\quotient{\backslash\negthickspace\sim}
\begin{document}
	Kusterskiy Dmitriy
	
	Lapin Andrew
	
	Markeeva Larisa
	
	Usvyatsov Mikhail
	
	
	\centerline{\LARGE Theorem 1}

	\bigskip
			
If h(n) is consistent, A* using GRAPH-SEARCH is optimal
\bigskip

Solution

Suppose that the algorithm chose non optimal path in the graph on the picture. This path goes through the node A. Whereas the optimal path goes through the node B. 
Let us consider the common node for optimal path and non optimal path named begin. Furthermore end node is the union point of these paths.

\begin{center}
	\begin{tikzpicture}[scale=.8,auto=left,every node/.style={circle,fill=blue!20}]
  \node (n0) at (7,9) {Start};
  \node (n1) at (7,6){begin};
  \node (n2) at (7,-3){end};
  \node (n3) at (1,1.5){A};
  \node (n4) at (7,3){B};
  \node (n5) at (7,0){C};
  \node[draw=none, rectangle, fill=none] (n7) at (7,4.5){$\vdots$};
  \node[draw=none, rectangle, fill=none] (n7) at (7,1.69999){$\vdots$};
  \node[draw=none, rectangle, fill=none] (n8) at (3.5,3.5){$\iddots$};
  \node (n9) at (12,-3){GOAL};

  \foreach \from/\to in {n2/n3,n2/n5,n2/n9,n0/n1}
    \draw (\from) -- (\to);

\end{tikzpicture}
\end{center}		

Imagine that the algorithm is in node A. And Node B cost is bigger than A's cost. But C node cost is less than A's node cost. So Path trough C node is optimal. Then:

f(B) = g(B) + h(B)

f(A) = g(A) + h(A)

f(C) = g(C) + h(C)

f(A) < f(C) Because the cost though C is less than though A. 

f(B) > f(A) Because the cost of B node is bigger than C's cost

\[\left\{
\begin{array}{l}
  g(B) + g(B \rightarrow C) + h(C) < g(A) + h(A), \text{it is because the path through A node is not optimal}\\
  g(A) + h(A) < g(B) + h(B) \ 
\end{array} \right. \]

Than $g(B \rightarrow C) < h(B) - h(C)$

That breaks consistency condition. So algorithm had to go through B node with less cost than the cost of A node. This solution is always optimal that is why GRAPH-SEARCH with $A^*$ is optimal

QED
\end{document}
