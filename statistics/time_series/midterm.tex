\documentclass[]{article}

%opening
\title{Midterm}
\author{Usvyatsov Mikhail}

\begin{document}

\maketitle

\section{Part 1}

\begin{enumerate}
	\item
		We know, that the first autocorrelation coefficient $\phi_1$ is close to one and so we can write that $y_t=\phi_0+\phi_1 y_{t - 1}+\epsilon_t$. Due to the fact, that $\phi_1$ is close to one we can decide, that there is a unit root. So, Canadian inflation rate has a stochastic trend. One of more formal approach to test for the unit root is Augmented Dickey–Fuller test.
	\item
		
\end{enumerate}

\end{document}
