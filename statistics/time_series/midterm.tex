\documentclass[]{article}
\usepackage{enumitem}

%opening
\title{Midterm}
\author{Usvyatsov Mikhail}

\begin{document}

\maketitle

\section{Part 1}

\begin{enumerate}[label=(\alph*)]
	\item
		We know, that the first autocorrelation coefficient $\phi_1$ is close to one and so we can write that $y_t=\phi_0+\phi_1 y_{t - 1}+\epsilon_t$. Due to the fact, that $\phi_1$ is close to one we can decide, that there is a unit root. So, Canadian inflation rate has a stochastic trend. One of more formal approach to test for the unit root is Augmented Dickey–Fuller test.
	\item
		$H_0: \theta = 0$ \\
		$ADF = - \frac{0.1}{0.05}$ = -2
	\item
		We have to accept Null hypothesis in all ocasions, because -2 is less negative than all the critical values.
	\item
		It is necessary to look on the correlation between $lag_i$ and current value of series. If the correlation is statistically significant - we have to include this lag. 
	\item
		AR(1):\\
		$\delta Inf_t$ = 0.002 - 0.31 * (-1.5) =  0.46\\
		Inflation rate = $Inf_{t-1} + \delta Inf_t$ = 1.3 + 0.46 = 1.76\\
		The error is 1.76 - 2.1 = -0.34
		
		AR(4):\\
		$\delta Inf_t$ = 0.021 - 0.46 (-1.5) - 0.39 (-1.4) - 0.25 (3.5) = 0.382\\
		Inflation rate = $Inf_{t-1} + \delta Inf_t$ = 1.3 + 0.382 = 1.682\\
		The error is 1.682 - 2.1 = -0.418
		
		ADL(4,1):\\
		$\delta Inf_t$ = 1.279 - 0.51 (-1.5) - 0.44 (-1.4) - 0.3 (3.5) - 0.16 (7) = 0.49\\
		Inflation rate = $Inf_{t-1} + \delta Inf_t$ = 1.3 + 0.49 = 1.79\\
		The error is 1.79 - 2.1 = -0.31
	\item
\end{enumerate}

\end{document}
