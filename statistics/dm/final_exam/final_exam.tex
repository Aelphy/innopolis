\documentclass[12pt]{article}

\usepackage[utf8]{inputenc}
\usepackage[T2A]{fontenc}
\usepackage[english,russian]{babel}
\usepackage{amssymb}
\usepackage{graphicx}
\graphicspath{ {images/} }

\textwidth=431pt
\textheight=600pt
\hoffset=-30pt
\voffset=-30pt

\usepackage{graphicx}
\usepackage{amsmath}
\makeatletter
\renewcommand{\@oddhead}{%
\vbox{%
\hbox to \textwidth{\strut \textit{Decision Making, Final exam, Usvyatsov Mikhail} \hfill }
%\hbox to\textwidth{Лист\hfill Страница~\arabic{page}~из 2}
\hrule
\vspace{12pt}
}}
\renewcommand{\@oddfoot}{}
\makeatother

\begin{document}

%\tableofcontents

%\newpage

\begin{center}
\textbf{Take home exam;\\
Due: January 7}
\end{center}

\bigskip
	
\textbf{Exercise 1}		
\newcounter{bcounter}
\begin{list}{(\alph{bcounter})~}{\usecounter{bcounter}}
\item 
We know that:
$\int_{-\infty}^{\infty} PDF(x) dx = 1$, as far that we know that the 0 < x < 4, we can change integration limits, so $\int_{0}^{4} \dfrac{c}{\sqrt{x}} dx = 1$.
Finally we can conclude that: \\
 с = $\dfrac{1}{4}$
\item
$CDF(x) = \int_{-\infty}^{x} PDF(x) dx$\\
CDF(x) = 
$\begin{cases}
0, x < 0 \\
\dfrac{\sqrt{x}}{2}, 0 \leqslant x \leqslant 4 \\
 1, x > 4
\end{cases}$
\item
P(x < 0.25) = $CDF(0.25) = 0.25$\\
P(x > 1) = $1 - CDF(1) = 0.5$\\
\item
$Y = \sqrt{X}$\\
$F_Y(y) = P(Y < y) = P(\sqrt{X} < y) = P(X < y^2) = CDF(y^2)$\\
$F_Y(y) = 
\begin{cases}
0, y < 0 \\
\dfrac{y}{2}, 0 \leqslant y \leqslant 2 \\
 1, y > 2
\end{cases}$
\item
$E(y) = \int_{-\infty}^{\infty} y \cdot PDF(y) dy$\\
$E(y) = \int_{0}^{2} \dfrac{y}{2} dy = 1$\\
$Var(y) = E(y^2) - E^2(y) = \int_{0}^{2} \dfrac{y^2}{2} dy - 1 = \dfrac{1}{3}$
\end{list}
\medskip

\textbf{Exercise 2}

\newcounter{ccounter}
\begin{list}{(\alph{ccounter})~}{\usecounter{ccounter}}
\item
We know from the CLT that, $\sqrt{n}\left(\dfrac{1}{n} \sum^{n}_{i = 1} X_i - Ex\right) \sim N(0,\sigma^2)$, so we can derive, that: $\overline{X} \sim N\left(Ex, \dfrac{\sigma^2}{n}\right)$\\

From the initial distribution we can find Ex and $\sigma^2$:

PDF(x) = $\dfrac{d F(x)}{dx} = 4 x^{-5}$\\
$E(x) = \int_{-\infty}^{\infty} x \cdot PDF(x) dx = \int_{1}^{\infty} \dfrac{4}{x^{-4}} dx = \dfrac{4}{3}$ \\
$\sigma^2 = Ex^2 - E^2x = \int_{1}^{\infty} \dfrac{4}{x^{-3}} dx = \dfrac{2}{9}$, and so:\\
$\overline{X} \sim N\left(\dfrac{4}{3}, \dfrac{2}{9 \cdot n}\right)$\\
\item
Let g(x) = ln(x)\\

We can use delta method, because $g'(Ex) \neq 0$ and g(x) has a derivative equals to $\dfrac{1}{x}$

As far as we know that:\\
$\sqrt{n}\left(\overline{X} - \dfrac{4}{3}\right) \sim N(0,\sigma^2)$\\
We can conclude from delta method:\\
$\sqrt{n} \left(g(\overline{X}) - g\left(\dfrac{4}{3}\right)\right) \sim N\left(0, \sigma^2 \left[ g'\left(\dfrac{4}{3}\right) \right]^2\right)$, and so:\\
$ln\left(x\right) \sim N\left(ln\left(\dfrac{4}{3}\right), \dfrac{1}{8 \cdot n}\right)$
\item
We know that: \\
$\dfrac{3 \cdot \sqrt{n}}{\sqrt{2}} \left(\overline{X} - \dfrac{4}{3}\right) \sim N(0,1)$\\
Thus, $\dfrac{9 \cdot n}{2}\left(\overline{X} - \dfrac{4}{3}\right)^2 \sim \chi_1^2$\\
$n \cdot \left(\overline{X} - \dfrac{4}{3}\right)^2 \sim \dfrac{2}{9} \cdot \chi_1^2$\\
$n \cdot \left(\overline{X} - \dfrac{4}{3}\right)^2 = n \cdot \left(\overline{X} - 0.8 + 0.8 - \dfrac{4}{3}\right)^2 = n \cdot \left(\overline{X} - 0.8\right)^2 - 2 \cdot \dfrac{8 \cdot n}{15} \cdot \left(\overline{X} - 0.8\right) + \dfrac{64}{15^2}$\\
$2 \cdot \dfrac{8 \cdot n}{15} \cdot \left(\overline{X} - 0.8\right) \sim N\left(\dfrac{32}{15}, \dfrac{15^2}{16^2 \cdot n^2}\right)$\\
So, $n \cdot \left(\overline{X} - 0.8\right)^2 \sim \dfrac{2}{9} \cdot \chi_1^2 + N\left(\dfrac{416}{225}, \dfrac{15^2}{16^2 \cdot n^2}\right)$
\end{list}

\textbf{Exercise 3}

\begin{list}{(\alph{ccounter})~}{\usecounter{ccounter}}
\item
\begin{enumerate}
\item
From Chebyshev-Markov inequality we know that:\\
$P\left( \vert \overline{X} - \mu \vert > \varepsilon \right) \leq \dfrac{\sigma^2}{n \cdot \varepsilon^2}$\\
We can find that $\varepsilon = 1.4$ and so, the probability that $\overline{X}$ is not in interval less than 25 \%
\item
From CLT we know that $\overline{X} \sim N\left( \mu, \dfrac{\sigma^2}{n} \right)$\\
To find the probability that $\overline{X}$ is not in the interval we have to find the value of Laplass function.\\
$P\left( \vert N(128, \dfrac{6.3^2}{81}) - \mu   \vert > \varepsilon \right) = 2 \cdot \Phi\left(\dfrac{\mu + \varepsilon}{\sigma^2}\right)$ = 0.02394\\
And so, the probability that $\overline{X}$ in not in the interval is 1 - 0.02394 = 97.6\%
\end{enumerate}
\item
$129 \pm 1.96 \cdot 6.3 = [116.652, 141.348] $
\end{list}
\medskip

\medskip		

\textbf{Exercise 4}

\begin{list}{(\alph{ccounter})~}{\usecounter{ccounter}}
\item
\item
\item
\end{list}
\medskip

\textbf{Exercise 5}

\begin{list}{(\alph{ccounter})~}{\usecounter{ccounter}}
\item
\item
\item
\end{list}

\medskip

\textbf{Exercise 6}

\begin{list}{(\alph{ccounter})~}{\usecounter{ccounter}}
\item
\item
\item
\end{list}

\end{document}